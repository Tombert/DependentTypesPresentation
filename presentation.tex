\documentclass{beamer}
\title{An Intro to Dependent Types with Idris}
\subtitle{}
\author{Thomas Gebert}
\date{August 9, 2016}
\usefonttheme{serif}
\usepackage{listings}
\usepackage{color}
\definecolor{lightpurple}{rgb}{0.8,0.8,1}

\definecolor{mygray}{rgb}{0.5,0.5,0.5} 
\lstset{ %
  backgroundcolor=\color{white}, 
  basicstyle=\footnotesize,       
  breakatwhitespace=false,        
  breaklines=true,                 
  captionpos=b,                    
  commentstyle=\color{mygray},   
  escapeinside={\%*}{*)},        
  extendedchars=true,              
  frame=single,                  
  keywordstyle=\color{blue},       
  language=Haskell,                
  numbers=left,                    
  numbersep=5pt,                   
  numberstyle=\tiny\color{mygray},
  rulecolor=\color{black},        
  showspaces=false,               
  showstringspaces=false,          
  showtabs=false,                  
  stepnumber=2,                    
  stringstyle=\color{mymauve},   
  tabsize=2,                      
  title=\lstname,                  
  morekeywords={not,\},\{,preconditions,effects },            
  deletekeywords={time}            
}

\usetheme{Warsaw}

\begin{document}
  \frame{\titlepage}
\begin{frame}[fragile]{Problems with regular type systems}
	
 	\transdissolve
	Haskell, F\#, and OCaml all have great type systems, but still there are still holes that can be improved on.
	\pause
	\begin{lstlisting}
		head []
		-- *** Exception: Prelude.head: empty list

		printf "blah %s %s" "hello"
		-- *** Exception: printf: argument list ended 
    -- prematurely
	\end{lstlisting}

\end{frame}
  \begin{frame}{What causes these problems?}
 		\transdissolve
	There are two main reasons that this happens
	  \begin{itemize}
		\pause
		\item Most functional languages have two languages that can't interact: the Type language, and the programming language \begin{itemize}
			\pause
			\item Types exist only as an enforcement layer
			\end{itemize}
		\pause
		\item Since these languages can't interact, all the types and conditions for them must be known by the programmer ahead of time, and can't can't be deduced from the context of the code.
	  \end{itemize}
  \end{frame}
\begin{frame}
\begin{center}
 		\transdissolve
 The solution to these problems (and many others) are dependent types
\end{center}
\end{frame}

\begin{frame}{So what the hell is a dependent type? }
 	\transdissolve
	A dependently typed language generally means two things
	\begin{itemize}	
		\pause
		\item Types, like functions, are first-class citizens that can be built dynamically (without it being dynamic typing)
		\pause
		\item Types (return types, input types, etc) can change depending on values
		\pause
                \item Functions can be called inside the type signature
	\end{itemize}
	\pause
	Common languages for it are: 
	\begin{itemize}	
		\item Agda
		\item Coq 
		\item F*
	\end{itemize}
\end{frame}
\begin{frame}{Enter Idris}
	\transdissolve
	What is Idris? 
	\begin{itemize}
		\pause
		\item Idris is a Haskell-like language by Edwin Brady \begin{itemize}
				\item Similar syntax, but not lazy
	\end{itemize}
	\pause
\item Dependent types are a main feature, but less dogmatic than Agda \begin{itemize}
		\item Long-term goals are dependently typed system drivers
	\end{itemize}
		\pause
		\item Outputs to C, LLVM, JavaScript, and PHP \begin{itemize}
				\item Making a new backend can usually be done in \textless 500 LOC
			\end{itemize}
	\end{itemize}
\end{frame}
\begin{frame}[fragile]{A safe list with a length}
	\transdissolve
	Lists are annoying for a couple reasons
	\begin{itemize}
		\pause
		\item The aforementioned issue with unexpected empty lists causes runtime errors
		\pause
		\begin{lstlisting} 
			head [] 
		  -- *** Exception: Prelude.head: empty list
		\end{lstlisting}
	\pause
	\item Even getting the length requires O(n) operations.\begin{itemize}
			\pause
			\item You could store the length as a property, but that requires anyone who updates length to make sure it's updated
		\end{itemize}
		
	\end{itemize}
	
\end{frame}
\begin{frame}{A safe list with a length}
	\transdissolve
	\begin{center} 
		Let's code it!
	\end{center}
\end{frame}

\begin{frame}{A safe printf (or any variadic function)}
	\transdissolve
	Variadic functions (in most langauges) are convenient, but terrible.
	\begin{itemize}
	  \pause
          \item Mixing types (like in printf) generally eschews compiler type-safety
	  \pause
	  \item Compiler can't check to see if you have the correct number of arguments
	\end{itemize}
\end{frame}
\begin{frame}{A safe printf (or any variadic function)}
	\transdissolve
	\begin{center} 
	  More coding!
	\end{center}
\end{frame}
\begin{frame}{A note about compiler hacks}
 		\transdissolve
  F\#'s printfn works as you would expect due to a special case in the compiler doing static analysis on that particular case. 
  \linebreak
  \linebreak
  \pause
  This is totes OK, but that only works for that particular case.
\end{frame}
\begin{frame}[fragile]{Proofs}
 	\transdissolve
	Idris (and most dependently typed languages) allows for mathematical proofs of program correctness. 
	\begin{itemize}
		\pause
		\item Quick example
			\begin{lstlisting}
				fiveIsFive : 5 = 5
				fiveIsFive = Refl
			\end{lstlisting}

		\pause
		\item Very cool but a bit beyond the scope of this talk \begin{itemize} 
				\pause
				\item (I'm still learning how to use them)
			\end{itemize}
	\end{itemize}
\end{frame}
\end{document}
